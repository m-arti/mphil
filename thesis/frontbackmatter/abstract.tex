%*******************************************************
% Abstract
%*******************************************************
%\renewcommand{\abstractname}{Abstract}
\pdfbookmark[1]{Abstract}{Abstract}
% \addcontentsline{toc}{chapter}{\tocEntry{Abstract}}
\begingroup
\let\clearpage\relax
\let\cleardoublepage\relax
\let\cleardoublepage\relax

\chapter*{Abstract}

Mis- and disinformation, commonly collectively called \emph{fake news}, continue to menace society. Perhaps, the impact of this age-old problem is presently most plain in politics and healthcare. However, fake news is affecting an increasing number of domains. It takes many different forms and continues to shapeshift as technology advances. Though it arguably most widely spreads in textual form, \emph{e.g.}, through social media posts and blog articles.

Thus, it is imperative to thwart the spread of textual misinformation, which necessitates its initial detection. This thesis contributes to the creation of representations that are useful for detecting misinformation.

Firstly, it develops a novel method for extracting textual features from news articles for misinformation detection. These features harness the disparity between the \emph{thematic coherence} of authentic and false news stories. In other words, the composition of themes discussed in both groups significantly differs as the story progresses.

Secondly, it demonstrates the effectiveness of topic features for fake news detection, using classification and clustering. Clustering is particularly useful because it alleviates the need for a labelled dataset, which can be labour-intensive and time-consuming to amass.

More generally, it contributes towards a better understanding of misinformation and ways of detecting it using Machine Learning and Natural Language Processing.

\vfill

\endgroup

\vfill
